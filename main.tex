\documentclass{article}
\usepackage{graphicx} % Required for inserting images
\usepackage[backend=biber, style=authoryear]{biblatex}
\usepackage{amssymb}
\usepackage{amsmath}

% ----- theorem-like environments -----
\newtheorem{theorem}{Theorem}
\newtheorem{lemma}{Lemma}
\newtheorem{proposition}{Proposition}
\newtheorem{corollary}{Corollary}

% ----- definition-like environments (not in italics) -----
\theoremstyle{definition}
\newtheorem{definition}{Definition}
\newtheorem{example}{Example}
\newtheorem{proof}{Proof}
% ----- remark-like environments (small text, non-italic) -----
\theoremstyle{remark}
\newtheorem{remark}{Remark}


\addbibresource{references.bib}
\begin{document}
\title{Coalitional Stability in Congestion Games with Altruism and Spite}
\author{Junxiang Li, Jiayi Chen, Yujie Xu, Yiqi Qiu}
\date{November 2025}
\maketitle

\section{Introduction}
\subsection{Fundamental Works on Congestion Games}
 Remark, definition这些要再生成
Congestion games form a foundational class of strategic interaction models where agents compete over shared, delay-sensitive resources. First introduced by \textcite{rosenthal1973class}, congestion games have become central to algorithmic game theory, network economics, and decentralized optimization. A standard congestion game models situations where each player selects a subset of shared resources, and the cost of each resource depends solely on the number of players choosing it. Rosenthal demonstrated that every finite congestion game admits at least one pure-strategy Nash equilibrium by constructing an exact potential function. This result marked one of the earliest applications of potential-based methods to equilibrium analysis.

Building on Rosenthal's insight, \textcite{monderer1996potential} formalized the notion of potential games, showing that every finite congestion game is an exact potential game. Their framework provided a unified theoretical foundation for numerous decentralized systems where agents’ incentives align with an underlying global function. These developments positioned congestion games as analytically tractable yet behaviorally rich models of strategic interaction.

A parallel line of inquiry has focused on the efficiency of equilibria in such games. \textcite{koutsoupias1999worst} introduced the concept of the Price of Anarchy (PoA)—the ratio between the cost of a worst-case Nash equilibrium and that of a socially optimal outcome. This benchmark enabled quantitative assessments of inefficiency under selfish behavior. In a seminal contribution, \textcite{roughgarden2002how} analyzed the PoA in nonatomic routing games with affine latency functions and established a tight bound of $4/3$. Their work not only provided explicit inefficiency guarantees but also introduced the smoothness framework that now underpins much of the performance analysis in algorithmic game theory.

These foundational contributions collectively defined the standard tools and benchmarks for analyzing strategic resource allocation. However, real-world applications often involve heterogeneous preferences, cooperation, and social context—elements not captured by purely selfish models. Consequently, a growing body of research seeks to extend the classical framework to incorporate social preferences (e.g., altruism, spite), coalition formation, and bounded rationality. Understanding the interplay between strategic behavior and social structure in these extended models remains a central challenge in the theory of network games.


\subsection{Congestion Games with Social Preferences: Altruism and Spite}

Classical congestion games model purely selfish agents who seek to minimize their own delays. However, many real-world systems involve \emph{social preferences}, where agents may also care about the impact of their actions on others. This behavior is commonly modeled via a weighted combination of individual and external delays:
\[
\mathrm{Perceived\ Cost} = (1 - \alpha) \cdot \mathrm{own\ delay} + \alpha \cdot \mathrm{others'\ delay}
\]
Here, $\alpha > 0$ captures \emph{altruism}, while $\alpha < 0$ represents \emph{spite} or \emph{hostility}—a desire to increase others’ costs.

\textcite{chen2008altruism} first introduced this formulation in traffic routing and derived tight bounds for the Price of Anarchy (PoA). In networks with convex latency functions, they showed that if all agents share altruism level $\beta > 0$, then
\[
\mathrm{PoA} \leq \frac{1}{\beta}, \quad \mathrm{and\ in\ linear\ networks,\ } \mathrm{PoA} \leq \frac{4}{3 + 2\beta - \beta^2}
\]

This bound remains valid even for $\beta < 0$, meaning that spiteful preferences also admit a symmetric analysis, though often with degraded efficiency.

\textcite{caragiannis2010altruism} extended the model to atomic congestion games with uniform altruism, proving that in linear latency settings, PoA increases with $\alpha$:
\[
\text{PoA} = \frac{5 + 4\alpha}{2 + \alpha}
\]
This demonstrates that altruism can, perhaps counterintuitively, worsen equilibrium outcomes in atomic games. However, in symmetric load balancing games, they showed that moderate altruism can restore optimality.

\textcite{chen2014robust} introduce the smoothness framework as a unified
tool for bounding the robust Price of Anarchy (PoA) in both atomic and
nonatomic congestion games, demonstrating that smoothness guarantees extend
simultaneously to pure, mixed, correlated, and coarse correlated equilibria.
Building on this foundation, \textcite{chen2014altruism} identify a striking
dichotomy: altruism deteriorates PoA in atomic environments but strictly
improves PoA in nonatomic (Wardrop) settings. Moreover, they further show
that in symmetric singleton congestion games with linear latency functions,
the pure PoA decreases monotonically as the average altruism level increases.


\textcite{hoefer2009altruistic} examined equilibrium existence under altruism. While pure equilibria persist in symmetric singleton games with convex delays, the game may lack equilibria under concave latency. Furthermore, they proved that determining equilibrium existence is NP-hard in general, suggesting that social preferences can break the finite improvement property characteristic of classical congestion games.

On the spiteful side, \textcite{kleer2017perception} proposed a parameterized perception model, where agents perceive others’ congestion scaled by $\rho$, and social cost is evaluated with scale $\sigma$. Here, $\rho > 1$ captures altruism, while $\rho < 1$ captures spite. They derived tight bounds on PoA and PoS for affine latency games over all $(\rho, \sigma)$ combinations, showing that tuning perception can significantly affect inefficiency.

\textcite{schroder2020spite} extended this framework by establishing upper bounds on PoA and PoS under high spite (very negative $\rho$), thus answering open questions from Kleer and Schäfer. His results confirm that strong hostility leads to maximal inefficiency, further motivating refined models for social preferences in routing and resource games.

Other generalizations explore diverse social structures:\textcite{bilo2013} models heterogeneous altruistic preferences using a 
global social context matrix $\Gamma$, where each player $i$ evaluates a weighted combination of all players' true costs $\tilde{c}_i(s)=\sum_j \gamma_{ij}c_j(s)$. Under nonnegative, restricted, and symmetric $\Gamma$, the resulting congestion game remains an exact potential game, ensuring pure Nash equilibrium existence and bounded inefficiency.; \textcite{kuznetsov2010range} incorporate friend-oriented utility into 
network formation by modifying each player's objective to combine self-payoff 
with the payoffs of an exogenously defined friendship set, and analyze resulting 
equilibria under multiple stability notions (unilateral, pairwise, and coalition-based). 
They show that such relational preferences can fundamentally change stable 
network outcomes, allowing dense and cohesive community structures to emerge 
as equilibrium networks that would be unstable under purely selfish behavior; and \textcite{iwan2010fairness} extend classical load balancing models by assuming that players are not purely selfish but also care about fairness in how load is distributed across machines or resources. Instead of minimizing only their individual latency, players incorporate a fairness penalty term—such as envy, load inequality, or deviation from balanced allocation—into their perceived cost. This modification changes equilibrium behavior: players may choose strategies that result in more balanced and less envious allocations, potentially sacrificing small amounts of efficiency to avoid unfair or highly skewed outcomes.

These contributions reveal a nuanced landscape: altruism and spite can both enhance and destabilize equilibrium outcomes, depending on the game structure, latency properties, and player heterogeneity. Effective modeling requires careful attention to how social preferences interact with congestion dynamics and strategic incentives.

\subsection{Coalition Formation in Congestion Games}

While classical congestion games assume non-cooperative players, several extensions explore the impact of cooperation through \emph{coalition formation}. In such settings, groups of players jointly choose strategies to optimize collective payoffs, leading to new equilibrium concepts and structural properties.

\textcite{fotakis2006atomic} introduced coalition-based atomic congestion games, modeling each coalition as a meta-player optimizing total latency across its members. For linear latency functions, they showed that these games admit exact potential functions and thus pure-strategy Nash equilibria still exist. Interestingly, the Price of Anarchy (PoA) remains the same on identical parallel-link networks unless coalitions grow very large. However, they also identified increased computational complexity: finding a coalition equilibrium becomes PLS-hard, even if best-response dynamics retain convergence under certain cost conditions.

\textcite{kuniavsky2014coalitional} further formalized coalitional congestion games and analyzed when such games preserve potential structure. They established necessary and sufficient conditions for equilibrium existence based on coalition structures and strategy sets. Their findings suggest that in many cases, forming coalitions destroys the potential function, making equilibrium guarantees less robust.

In the nonatomic setting, \textcite{wan2012composite} considered a composite model blending individuals and coalitions. He proved that coalition formation strictly benefits participants: at equilibrium, each coalition’s average cost and each individual's cost do not exceed their costs in the non-cooperative game. Moreover, larger coalitions achieve lower average delay than smaller ones—yet individual players fare best—highlighting complex efficiency trade-offs in such systems.

The \emph{Price of Collusion (PoC)} was introduced by \textcite{hayrapetyan2006collusion} to quantify how much collusion improves or worsens outcomes. For symmetric load-balancing games with convex latency, they showed $\mathrm{PoC} \leq 2$, but in asymmetric settings PoC can grow arbitrarily large. Notably, they demonstrated that in nonatomic games with splittable flows, collusion cannot harm social cost ($\mathrm{PoC} = 1$), contrasting sharply with atomic cases.

More recently, \textcite{sultana2024coalition} proposed a partition-form model where each coalition’s payoff depends on how the rest of the players group themselves. Assuming inter-coalition Nash play, they characterized stable coalition structures. Surprisingly, they found that grand coalitions (full cooperation) are rarely stable unless extreme conditions hold—such as one overwhelmingly dominant resource. Instead, typical equilibria feature only small coalitions or none at all, sometimes resulting in cyclic re-formations. Their findings reveal that coordination costs and structural asymmetries can prevent efficient coalitional outcomes, even when social welfare would be improved.

These studies show that allowing coalition formation introduces richer dynamics, new computational challenges, and the need for refined equilibrium concepts like strong equilibria, coalition-proof Nash, or partition equilibria. Furthermore, the interaction of coalitional behavior with social preferences (e.g., altruism or spite) opens further complexity: cooperation can amplify or mitigate inefficiencies depending on the network, latency function, and agents’ incentives.
(可以加一些社会学的研究,ingroup outgroup的倾向,让讨论更方针——
\subsection{Research Gap and Thesis Statement}
The existing literature studies three well-develope yet fundamentally independent research fields: (i) classical congestion games based on potential theory, (ii) social-preference extensions that model altruism and spite, and (iii) coalition formation and cooperative behavior in resource networks. The classic congestion models assume fully selfish players, and use potential functions to support the Nash equilibria. Studies on altruism and spite usually focus on fully non-cooperative agents and equilibrium efficiency measures such as price of anarchy. Meanwhile, research on coalitional congestion games often treats players as purely rational and self-interested and neglects how social concerns could reshape cooperation itself. 

Although each direction has been studied extensively, their intersection of how social preferences influence the formation and stability of coalitions remains understudied. In particular, we are missing a clear understanding of how coalition-based social preferences, specifically in-group altruism and out-group spite, shape the coalition in congestion games. Additionally, many prior researches assume perfect rational behavior and complete information, overlooking factors such as the learning noise and imperfections of feedback, which is common in realistic systems. Such assumption lead to the problem of neglecting the impact of learning noise and imperfect feedback on the outcomes of the games.  


\section{Preliminaries}

We study a congestion game in which players may form coalitions and evaluate outcomes using social preferences that combine in-group altruism and out-group spite. These preferences are parameterized by an internal weight $\rho$ and an external weight $\sigma$, and they determine how coalition members aggregate one another’s payoffs and the payoffs of players outside the coalition. Coalition structures are not fixed: they may form, break, or reorganize endogenously. Our central question is therefore not whether selfish Nash equilibria remain unchanged, but rather for given values of $(\rho,\sigma)$ which coalition structures remain stable—i.e., no coalition has an incentive to deviate, split, or merge—and whether the induced game continues to preserve potential-game guarantees under such coalition deviations.

\subsection{Congestion basics}
\label{subsec:congestion-basics}

Let $N=\{1,\dots,n\}$ be a finite set of players, and $K=\{1,\dots,m\}$ a finite set of resources. 
Each player selects exactly one resource.  
A strategy profile is $s=(s_1,\dots,s_n)\in K^n$, where $s_i\in K$ denotes the resource chosen by player $i$.

For each resource $k\in K$ and profile $s$, define the congestion level
\[
n_k(s) \;=\; \bigl|\{\, i\in N : s_i = k \,\}\bigr|.
\]

Following the standard setting of linear congestion games, each resource $k$ is equipped with a 
\emph{linear latency function}  
\[
f_k(n) \;=\; \alpha_k n + \beta_k,
\qquad \alpha_k,\beta_k \ge 0,
\]
where the latency increases linearly with congestion.

If player $i$ selects resource $k$ under profile $s$, her latency is
\[
\ell_i(s) \;=\; f_{s_i}\!\bigl( n_{s_i}(s) \bigr)
= \alpha_{s_i}\, n_{s_i}(s) + \beta_{s_i}.
\]

For convenience, we define the selfish payoff
\[
\mu_i(s) := -\ell_i(s),
\]
so that higher payoff corresponds to lower latency.
Under purely selfish behavior, maximizing $\mu_i$ yields the standard congestion game.


\subsection{Altruism and Spite}
\label{subsec:altruism-spite}

Players may also care about the performance of others, particularly those inside or outside their coalition. We model this using a normalized social-preference operator that averages the payoffs of group members. A coalition structure (defined formally in Section~\ref{subsec:coalition-structure}) partitions $N$ into disjoint coalitions. Let $P$ denote such a partition, and let $C_i(P)$ denote the coalition containing $i$.

Given a strategy profile $s$ and coalition structure $P$, the utility of player $i$ has three components:
(i) own payoff $\mu_i(s)$;  
(ii) a weight $\rho$ assigned to the \emph{average} payoff of in-group members $C_i(P)\setminus \{i\}$;  
(iii) a weight $-\sigma$ assigned to the \emph{average} payoff of out-group members $N\setminus C_i(P)$.

Formally,
\begin{equation}
    U_i(s;P)
    \;=\;
    \mu_i(s)
    \;+\;
    \rho \cdot \bar{\mu}_{\mathrm{in}}(i)
    \;-\;
    \sigma \cdot \bar{\mu}_{\mathrm{out}}(i),
    \label{eq:Ui}
\end{equation}
where the average payoffs are defined as:
\[
\bar{\mu}_{\mathrm{in}}(i) =
\begin{cases}
\dfrac{1}{|C_i|-1} \displaystyle\sum_{j \in C_i \setminus \{i\}} \mu_j(s), & |C_i| > 1 \\[12pt]
0, & |C_i| = 1
\end{cases}
\]
\[
\bar{\mu}_{\mathrm{out}}(i) =
\begin{cases}
\dfrac{1}{n - |C_i|} \displaystyle\sum_{m \notin C_i} \mu_m(s), & |C_i| < n \\[12pt]
0, & |C_i| = n
\end{cases}
\]

The parameters $\rho,\sigma\in\mathbb{R}$ are fixed and shared by all players.  
The interpretation is: $\rho>0$ represents in-group altruism, 
$\sigma>0$ represents out-group spite, and $\rho=\sigma=0$ yields the classical selfish model.

\medskip
\paragraph{Normalization rationale.}
This normalized formulation ensures that the social preference terms are scale-invariant with respect to coalition size. Under the previous sum-based formulation, utility would scale with coalition cardinality, causing large coalitions to experience disproportionately strong social effects. By averaging, we ensure that $\rho$ and $\sigma$ have consistent interpretations across different coalition structures: they represent the marginal rate of substitution between own payoff and the \emph{typical} ally's (or rival's) payoff, independent of group size.

\medskip
\paragraph{Boundary cases.}
When player $i$ is a singleton ($|C_i|=1$), there are no in-group members, so $\bar{\mu}_{\mathrm{in}}(i)=0$. When player $i$ belongs to the grand coalition ($|C_i|=n$), there are no out-group members, so $\bar{\mu}_{\mathrm{out}}(i)=0$. These boundary conditions ensure well-defined utilities at the extremes of coalition structure.

\subsection{Potential Function of the Underlying Congestion Game}
\label{subsec:potential-function}

Before introducing coalitional incentives, it is useful to recall the potential structure 
of the underlying selfish congestion game.  
The game described in Section~\ref{subsec:congestion-basics} is an exact potential game 
in the classical sense of Rosenthal, with potential
\begin{equation}
    \Phi^0(s)
    \;:=\;
    \sum_{k\in K}
    \sum_{x=1}^{n_k(s)} f_k(x),
    \label{eq:rosenthal-potential}
\end{equation}
where $f_k(\cdot)$ is the latency function of resource $k$.
If a single player $i$ changes her action while all others remain fixed, then
\[
\Phi^0(s_i',s_{-i}) - \Phi^0(s)
    = \ell_i(s_i',s_{-i}) - \ell_i(s),
\]
so any unilateral cost-improving deviation strictly decreases $\Phi^0$.  
Thus the pure Nash equilibria of the selfish game coincide with the local minima of 
$\Phi^0$.
\subsection{Preservation and Limitations of the Potential-Game Structure}

The social-preference operator $\Gamma(P)$ introduced in
Section~\ref{subsec:altruism-spite} satisfies
\[
\gamma_{ii}(P)=1,
\qquad 
\gamma_{ij}(P)=\gamma_{ji}(P)
\quad (i\neq j),
\]
for every coalition structure $P$.  
Symmetry and unit diagonal are therefore always preserved.
However, the entries $\gamma_{ij}(P)$ may be negative whenever
$\sigma>0$, corresponding to out--group spite.

Bil\`o's Theorem~\ref{thm:bilo}(3.1) guarantees the existence of an exact
potential function only under the stronger \emph{restricted altruistic}
assumptions of his model, namely:
\[
\gamma_{ij} \ge 0
\quad\text{and}\quad 
\sum_{j} \gamma_{ij} = 1
\quad\text{for all } i,
\]
in addition to symmetry.  
When these nonnegativity and normalization constraints hold, the game
induced by $(U_i(\cdot;P))_{i\in N}$ is an exact potential game, and a
pure Nash equilibrium exists.

In our framework, the condition $\gamma_{ij}(P)\ge 0$ is equivalent to
requiring $\sigma=0$, i.e.\ the absence of spite toward out--groups.
Thus Bil\`o's exact-potential result applies \emph{only} on the
altruistic slice
\[
(\rho,\sigma)\in \{\,(\rho,0): \rho \ge 0\,\}.
\]

For general social-preference parameters $(\rho,\sigma)$—in particular
whenever $\sigma>0$ introduces negative off-diagonal terms—the
restricted-altruism conditions are violated and Bil\`o's construction no
longer applies.  In these regions we cannot rely on exact potential
arguments to guarantee the existence of pure Nash equilibria; instead we
turn to the weaker notion of \emph{potential compatibility} developed in
Section~\ref{subsec:potential-function}, where Rosenthal’s potential
$\Phi^0$ acts as a Lyapunov function for small $(\rho,\sigma)$.




\subsection{Coalition Structure and Stability}
\label{subsec:coalition-structure}

A \emph{coalition structure} is a partition 
\[
P=\{C^1,\dots, C^r\}
\]
of the player set $N$ into pairwise disjoint coalitions whose union is $N$.
Two extreme cases will be important throughout:
\begin{itemize}
    \item the \emph{singleton partition} 
    $P^{\mathrm{sing}} = \bigl\{\{1\},\dots,\{n\}\bigr\}$,
    where every player acts independently; 
    \item the \emph{grand coalition} 
    $P^{\mathrm{grand}}=\{N\}$,
    where all players coordinate jointly.
\end{itemize}
Any intermediate partition $P$ represents a configuration in which some players form 
coordinating groups while others remain separate.

\paragraph{Coalition utilities.}
Given a coalition structure $P$ and a strategy profile $s$, 
each player’s utility is given by~\eqref{eq:Ui}.
The utility of a coalition $C\in P$ is the aggregate utility of its members:
\[
U_C(s;P) := \sum_{i\in C} U_i(s;P).
\]
A coalition behaves as a single decision-making entity and chooses the joint actions 
of all its members to maximize $U_C(\cdot;P)$.

\paragraph{Deviation by a coalition.}
For a fixed coalition structure $P$, a strategy profile $s^\ast$ is a 
\emph{coalitional Nash equilibrium} if no coalition can profitably change 
its joint action.  
Formally, for every $C\in P$ and every alternative joint deviation $s'_C$,
\[
U_C(s^\ast;P)\;\ge\;U_C(s'_C, s_{-C}^\ast;P).
\]
This notion holds the partition $P$ fixed and considers only deviations in
strategy space.

\medskip
\paragraph{Structural deviations (split and join).}
In many environments, players may freely enter or exit coalitions, and 
coalitions may merge or split.  
Thus we must also consider deviations in the \emph{coalition structure} itself.

\begin{itemize}
  \item \textbf{Split deviation.}
  A coalition $C^{ab}\in P$ may split into two subcoalitions
  $C^a$ and $C^b$ (a refinement of $C$), thereby inducing a new
  partition $P'$.  
  The split is profitable if the total utility of the resulting
  subcoalitions strictly exceeds the utility of the original coalition:
  \[
  U_{C^a}(s;P') + U_{C^b}(s;P')
  \;>\;
  U_{C^{ab}}(s;P).
  \]

  \item \textbf{Join deviation.}
  Two coalitions $C^a, C^b\in P$ may merge into a single coalition
  $C^{ab}=C^a\cup C^b$, producing a coarser partition $P'$.  
  The join is profitable if the utility of the merged coalition strictly
  exceeds the sum of the utilities of its constituents:
  \[
  U_{C^{ab}}(s;P')
  \;>\;
  U_{C^a}(s;P) + U_{C^b}(s;P).
  \]
\end{itemize}

These structural deviations represent a coalition's ability to 
freely leave a group, join another group, or merge for coordinated action. *In our later study, we assume every coalition member will only consider structural deviations after the coalitions reach the coalitional Nash equilibrium. 

\medskip
\paragraph{Coalition Equilibrium (New Stability Concept).}
Because players may revise \emph{both} their joint actions and their group 
memberships, we define the following notion of stabilityity.
\begin{definition}[Coalition Equilibrium]
A pair $(P^\ast,s^\ast)$ consisting of a coalition structure and a
strategy profile is a \emph{coalition equilibrium} if the following two
conditions hold:

\begin{itemize}
\item[(i)] \textbf{Action stability (internal stability within $P^\ast$).}
Given the stable coalition structure $P^\ast$, the profile $s^\ast$ is
stable against any coalition-level deviation in actions:
\[
U_C(s^\ast;P^\ast)
\;\ge\;
U_C(s'_C, s_{-C}^\ast;P^\ast)
\quad\text{for all } C\in P^\ast \text{ and all joint deviations } s'_C.
\]
\item[(ii)] \textbf{Structural stability (no profitable split or join).}
The coalition structure $P^\ast$ is stable against any refinement or
coarsening.  That is, for every coalition $C\in P^\ast$:

\begin{itemize}
    \item \emph{No profitable split:}  
    for every refinement of $C$ into two subcoalitions $C^a,C^b$ inducing
    a partition $P'$,
    \[
    U_{C^a}(s^\ast;P') + U_{C^b}(s^\ast;P')
    \;\le\;
    U_{C^{ab}}(s^\ast;P^\ast).
    \]

    \item \emph{No profitable join:}  
    for every pair of coalitions $C^a,C^b\in P^\ast$ and every coarsening
    that merges them into $C^{ab}=C^a\cup C^b$ inducing a partition $P'$,
    \[
    U_{C^{ab}}(s^\ast;P')
    \;\le\;
    U_{C^a}(s^\ast;P^\ast)+U_{C^b}(s^\ast;P^\ast).
    \]
\end{itemize}



\end{itemize}

If either property fails—if some split/join is profitable, or if some
coalition can improve by changing its joint action—then
$(P^\ast,s^\ast)$ is not a coalition equilibrium.
\end{definition}

This stability notion captures precisely the idea that coalitions form, dissolve,
or merge only when doing so increases the utility of the participating players.
Under parameters $(\rho,\sigma)$ that encourage strong in-group altruism or 
out-group hostility, the resulting equilibrium partition may be a singleton 
partition, the grand coalition, or an intermediate factional structure.

A coalition structure that is stable in both senses admits neither 
incentives to reorganize the partition nor incentives for coordinated 
action changes.  
Our objective is to understand how the social-preference parameters 
$(\rho,\sigma)$ guide randomly formed coalitions toward such stable 
configurations and how altruism and spite shape the emergence of 
singleton partitions, grand coalitions, or intermediate stable partitions.

\section{Social-Preference Regimes}
\label{sec:rho-sigma-first}

In this section we begin from the social side of the model---altruism
and spite---and work \emph{backwards} toward the congestion
environment.  Rather than fixing the resource parameters first and
examining how $(\rho,\sigma)$ adapt, we instead identify qualitative
coalition patterns that different social-preference regimes should
generate in an abstract congestion game.  Only then do we ask what
kinds of resource environments make these regimes observable and
analytically tractable.

\begin{proposition}[A computable sufficient bound for potential-compatibility]
\label{prop:pc-bound}
Consider a congestion game with linear latencies 
$f_k(n)=\alpha_k n+\beta_k$, and let 
$\alpha_{\min}=\min_k \alpha_k>0$ and 
$\alpha_{\max}=\max_k \alpha_k$.
If the social-preference parameters satisfy
\[
\rho+\sigma \;\le\; \frac{\alpha_{\min}}{2 n^2 \alpha_{\max}},
\]
then $(\rho,\sigma)$ is potential-compatible: for every coalition $C$ 
and deviation $s'\to s$, 
\[
\Delta U_C(\rho,\sigma)\ge 0 
\quad\Longrightarrow\quad 
\Delta\Phi^0\le 0.
\]
\end{proposition}

\begin{proof}
Fix a coalition $C$ and deviation $s'=(s'_C,s_{-C})\to s$.  
Write
\[
S_{\mathrm{in}} := \sum_{i\in C} \Delta\mu_i, 
\qquad    
S_{\mathrm{out}} := \sum_{m\notin C} \Delta\mu_m,
\qquad
S_{\mathrm{tot}} := S_{\mathrm{in}} + S_{\mathrm{out}}.
\]

For linear latencies $f_k(n)=\alpha_k n+\beta_k$, the Rosenthal
potential satisfies
\[
\Delta\Phi^0 = - S_{\mathrm{tot}},
\]
so potential-compatibility requires that
\[
\Delta U_C \ge 0 \quad \Rightarrow \quad S_{\mathrm{tot}} \ge 0.
\]

Using the utility definition,
\[
\Delta U_C
= 
(1+\rho(|C|-1)) S_{\mathrm{in}}
\;-\;
\sigma |C|\, S_{\mathrm{out}} .
\tag{1}
\label{eq:uc-expand}
\]

\paragraph{Bounding the social-preference distortion.}
Each individual payoff change satisfies
\[
|\Delta\mu_i|
\;\le\;
\alpha_{\max}(n-1),
\]
because a player's load can change by at most $n-1$.  Hence
\[
|S_{\mathrm{in}}| \le |C|\,\alpha_{\max}(n-1) \le n^2\alpha_{\max},
\qquad
|S_{\mathrm{out}}| \le n^2\alpha_{\max}.
\tag{2}
\label{eq:sin-sout-bounds}
\]

The part of $\Delta U_C$ that comes from social preferences is therefore
bounded by
\[
|\rho|\,|C|(|C|-1)\,\alpha_{\max} n
+
|\sigma|\,|C|(n-|C|)\,\alpha_{\max} n
\;\le\;
(\rho+\sigma)\, n^2\,\alpha_{\max}.
\tag{3}
\label{eq:distortion-bound}
\]

\paragraph{Baseline scale of $S_{\mathrm{tot}}$.}
Whenever $S_{\mathrm{tot}}\neq 0$, at least one player's load changes by
one unit on some resource with slope at least $\alpha_{\min}$, implying  
\[
|S_{\mathrm{tot}}| \;\ge\; \alpha_{\min}.
\tag{4}
\label{eq:baseline}
\]

\paragraph{Preventing sign reversal.}
Suppose $(\rho+\sigma) \le \alpha_{\min}/(2n^2\alpha_{\max})$.
Combining \eqref{eq:distortion-bound} and \eqref{eq:baseline},
\[
\frac{|\text{distortion}|}{|S_{\mathrm{tot}}|}
\;\le\;
\frac{(\rho+\sigma) n^2\alpha_{\max}}{\alpha_{\min}}
\;\le\;
\frac12 .
\]
Thus the social-preference distortion in \eqref{eq:uc-expand} can change
the magnitude of $S_{\mathrm{tot}}$, but not its sign:
\[
\operatorname{sign}(\Delta U_C)
=
\operatorname{sign}(S_{\mathrm{tot}}).
\]

Hence  
\[
\Delta U_C\ge 0 \quad\Rightarrow\quad S_{\mathrm{tot}}\ge 0
\quad\Rightarrow\quad \Delta\Phi^0 \le 0.
\]
Thus $(\rho,\sigma)$ is potential-compatible.
\end{proof}



This result is deliberately conservative and not meant for calibration.
Its value is conceptual: it guarantees the existence of a small
neighborhood around $(\rho,\sigma)=(0,0)$ in which social preferences
remain a perturbation of the selfish game and cannot overturn the descent
directions of the classical potential.

\subsection{Speculative Regimes in $(\rho,\sigma)$}
\label{subsec:regimes-rho-sigma}

Players evaluate outcomes according to
\[
U_i(s;P)
=
\mu_i(s)
+\rho \sum_{j\in C_i(P)\setminus\{i\}} \mu_j(s)
-\sigma \sum_{m\notin C_i(P)} \mu_m(s),
\]
and coalitions maximize $U_C(s;P)=\sum_{i\in C}U_i(s;P)$.
Even without specifying resources, the pair $(\rho,\sigma)$ induces
distinct qualitative patterns of coalition formation.  We group these
patterns into four speculative regimes.
\paragraph{(1) Nearly selfish regime: $(\rho,\sigma)$ in the potential-compatible basin.}

Let 
\[
\mathcal{B}
:= 
\Bigl\{(\rho,\sigma)\in\mathbb{R}^2 
:\; \rho+\sigma \le \frac{\alpha_{\min}}{2 n^2 \alpha_{\max}} \Bigr\}
\subset\mathbb{R}^2
\]
denote the sufficient potential-compatible region established in
Proposition~\ref{prop:pc-bound}.  
If $(\rho,\sigma)\in\mathcal{B}$, then the perturbation introduced by
social preferences satisfies
\[
\bigl| U_i(s;P) - \mu_i(s) \bigr|
= O(\rho n + \sigma n)
\ll \Delta \Phi^0,
\]
so that unilateral and coalitional deviations preserve the descent
direction of Rosenthal's potential~$\Phi^0$.  
In this regime:

\begin{itemize}
    \item all coalition-improving deviations are potential-improving;
    \item coalitional equilibria coincide with the (Rosenthal) Nash 
    equilibria of the selfish congestion game;
    \item the coalition structure $P$ is strategically inert:
    $U_C(s;P)$ is $\varepsilon$-close to 
    $\sum_{i\in C}\mu_i(s)$ with $\varepsilon\to 0$ as
    $(\rho,\sigma)\to(0,0)$.
\end{itemize}

Hence $\mathcal{B}$ is a small convex neighborhood of the origin in
which congestion incentives fully dominate social preferences.
\paragraph{(2) Altruistic regime: $\rho$ large, $\sigma$ small.}

Fix constants $\bar\rho>\frac{\alpha_{\min}}{2 n^2 \alpha_{\max}}$ and $0<c_A<1$.
Define the altruistic region
\[
\mathcal{A}
:=
\Bigl\{(\rho,\sigma)\in\mathbb{R}^2:\; 
\rho \ge \bar\rho,\;
|\sigma| \le c_A\,\rho 
\Bigr\}.
\]
When $(\rho,\sigma)\in\mathcal{A}$, coalition utilities satisfy
\[
U_C(s;P)
=
\sum_{i\in C}\mu_i(s)
\;+\;
\rho\,\sum_{i\neq j\in C}\mu_j(s)
\;+\;
O(\sigma n),
\]
so the dominant marginal gain comes from enlarging $C$.  
Thus:
\begin{itemize}
    \item join-deviations strictly dominate split-deviations;
    \item coalition sizes monotonically increase under best responses;
    \item the grand coalition $P^{\mathrm{grand}}=\{N\}$ is the natural
    stable configuration whenever congestion does not explode.
\end{itemize}

\paragraph{(3) Spiteful regime: $\rho$ small, $\sigma$ large.}

Fix constants $\bar\sigma>\frac{\alpha_{\min}}{2 n^2 \alpha_{\max}}$ and $0<c_S<1$.
Define the spiteful region
\[
\mathcal{S}
:=
\Bigl\{(\rho,\sigma)\in\mathbb{R}^2:\; 
\sigma \ge \bar\sigma,\;
|\rho| \le c_S\,\sigma 
\Bigr\}.
\]
Now
\[
U_C(s;P)
=
\sum_{i\in C}\mu_i(s)
\;-\;
\sigma\sum_{i\in C}\sum_{m\notin C}\mu_m(s)
\;+\;O(\rho n),
\]
so the dominant marginal incentive is to minimize exposure to
out-groups.  
Therefore:
\begin{itemize}
    \item join-deviations dilute the coalition’s ability to target
    outsiders;
    \item split-deviations strictly dominate except in degenerate cases;
    \item stable coalition structures are refinements of the
    singleton partition.
\end{itemize}

\paragraph{(4) Factional regime: $\rho$ and $\sigma$ large and comparable.}

Fix constants $\bar r>0$ and
$0<\underline{\lambda}<\overline{\lambda}<\infty$.
Let
\[
\mathcal{F}
:= 
\Bigl\{(\rho,\sigma)\in\mathbb{R}^2:\; 
\rho\ge \bar r,\;\sigma\ge \bar r,\;
\underline{\lambda} \le \tfrac{\sigma}{\rho} \le \overline{\lambda}
\Bigr\}.
\]
In this band, both in--group and out--group terms scale as
\[
U_C(s;P)
\approx
\rho\,\Bigl(
\sum_{i\neq j\in C}\mu_j(s)
-
\lambda\sum_{i\in C}\sum_{m\notin C}\mu_m(s)
\Bigr),
\quad
\lambda:=\tfrac{\sigma}{\rho},
\]
so neither “join at all cost” nor “split at all cost” is optimal.  
Balancing the two forces yields:
\begin{itemize}
    \item stable coalitions of intermediate cardinalities;
    \item equilibrium partitions consisting of several nontrivial blocs;
    \item a sharply differentiated coalition landscape reminiscent of 
    political factions or competitive gangs.
\end{itemize}

\paragraph{(5) Phase-diagram structure.}

Up to the choice of thresholds
$(\bar\rho,\bar\sigma,\bar r,c_A,c_S,
\underline{\lambda},\overline{\lambda})$,
the $(\rho,\sigma)$-plane admits a natural coarse partition:
\[
\mathbb{R}^2
\approx
\mathcal{B}
\;\sqcup\;
\mathcal{A}
\;\sqcup\;
\mathcal{S}
\;\sqcup\;
\mathcal{F},
\]
corresponding respectively to:
\begin{enumerate}
    \item the potential-compatible basin (nearly selfish behavior);
    \item the altruistic region (grand-coalition behavior);
    \item the spiteful region (fragmentation into singletons);
    \item the factional band (multiple intermediate-sized blocs).
\end{enumerate}
These sets are defined independently of any resource structure; they
capture the pure social-preference geometry of the model.







\subsection{Designing a Resource Environment to Reveal These Regimes}
\label{subsec:choosing-resources}

With the qualitative regimes identified, the purpose of the resource
environment becomes clear: to create a congestion landscape in which the
strategic consequences of each regime become observable and distinct.

Two principles guide this design.

\begin{enumerate}
    \item \textbf{Avoid degenerate environments.}  
    If all resources are identical and congestion is mild, then altruism
    and spite move utilities without producing meaningful strategic
    trade-offs.  Split and join incentives collapse, and the phase
    diagram loses expressive power.

    \item \textbf{Use controlled heterogeneity.}  
    To differentiate between grand coalitions, fragmentation, and
    factions, resources must reward different coalition sizes.  
    We therefore seek:
    \begin{itemize}
        \item a ``tight'' resource: attractive for small groups but
        punitive for large coalitions (high $\alpha_k$, low $\beta_k$);
        \item a ``forgiving'' resource: tolerates large coalitions (low
        $\alpha_k$, higher $\beta_k$);
        \item optionally, a resource best suited for medium-sized groups.
    \end{itemize}
\end{enumerate}

This motivates a simple heterogeneous environment used in the
experimental sections.  For example:
\[
f_1(n) = 1\cdot n + 0,
\qquad
f_2(n) = 0.5\cdot n + 2,
\qquad
f_3(n) = 1.5\cdot n + 1.
\]

Such an environment ``reveals'' the social regimes:

\begin{itemize}
    \item In the nearly selfish region, players distribute according to
    congestion slopes; coalition structure has minimal effect.
    \item In the altruistic region, coalitions gravitate toward the
    forgiving resource, and the grand coalition becomes stable.
    \item In the spiteful region, the tight resource amplifies the
    incentive to avoid sharing, promoting fragmentation.
    \item In the factional band, resources differentially appeal to
    medium-sized groups, creating stable blocs.
\end{itemize}

Rather than adjusting $(\rho,\sigma)$ to fit the resource environment,
we reverse the logic: we begin with the coalition patterns implied by
social preferences and then design a resource environment that makes
these patterns visible, comparable, and analytically tractable.






\section{Computational Experiments}

To validate the theoretical predictions established in Section 3 and to explore the empirical behavior of coalitional congestion games under normalized social preferences, we conduct a series of numerical simulations. Our analysis focuses on four key aspects: (i) stability phase diagrams for archetype coalition structures, (ii) the mechanism by which spite incentivizes coalition splitting, (iii) efficiency analysis through the price of stability, and (iv) the fragility of generic coalition structures.

\subsection{Experimental Setup}

We consider atomic congestion games with $n=6$ players and $m=3$ resources. Each resource $k$ is assigned a heterogeneous linear latency function designed to reveal distinct regime behaviors:
\[
f_1(n) = 1.0n + 0, \quad f_2(n) = 0.5n + 2, \quad f_3(n) = 1.5n + 1.
\]
This configuration represents a ``tight'' resource (high congestion sensitivity), a ``forgiving'' resource (low sensitivity but high base cost), and an intermediate resource. The potential-compatibility bound from Proposition~\ref{prop:pc-bound} evaluates to approximately $\rho + \sigma \leq 0.0019$ for this setting.

Social preferences are parameterized by in-group altruism $\rho \in [0, 0.01]$ and out-group spite $\sigma \in [0, 0.01]$. We test three archetype coalition structures that represent canonical endpoints of the coalition formation spectrum:
\begin{itemize}
    \item \textbf{Grand Coalition}: $P^{\mathrm{grand}} = \{N\}$ (full cooperation)
    \item \textbf{Singleton}: $P^{\mathrm{sing}} = \{\{1\}, \{2\}, \ldots, \{6\}\}$ (complete atomization)
    \item \textbf{Factions}: $P^{\mathrm{fac}} = \{\{1,2,3\}, \{4,5,6\}\}$ (two opposing blocs)
\end{itemize}
Nash equilibria are computed via best-response dynamics with convergence tolerance $\epsilon = 10^{-9}$. Coalition stability is assessed by exhaustively checking all possible split and join deviations.

\subsection{Archetype Stability Phase Diagrams}

We first examine which archetype structures can ``survive'' (remain stable) across the $(\rho, \sigma)$ parameter space. For each of 121 parameter points on an $11 \times 11$ grid, we compute the coalitional Nash equilibrium and then test for structural stability against both split and join deviations.

\begin{figure}[htbp]
\centering
\includegraphics[width=0.95\textwidth]{figures/archetype_stability_phases.png}
\caption{Stability phase diagrams for three archetype coalition structures across the $(\rho, \sigma)$ parameter space. Green indicates stable configurations; red indicates unstable. The singleton structure dominates with 91.7\% stability, while grand coalition and factions achieve only 0.8\% stability each.}
\label{fig:archetype_phases}
\end{figure}

Figure~\ref{fig:archetype_phases} presents striking asymmetry in stability outcomes. The singleton structure achieves 91.7\% stability across the parameter space, confirming our theoretical prediction that atomized players face no internal coordination pressures and are protected against join deviations when spite is present. In contrast, both the grand coalition and factional structures achieve only 0.8\% stability---stable only at the origin $(\rho, \sigma) = (0, 0)$ where the game reduces to the classical selfish case.

This result has important implications: \emph{even infinitesimal amounts of spite are sufficient to destabilize cooperative structures}. The normalized utility formulation amplifies this effect because spite operates on average out-group payoffs, making it profitable for subgroups to split off and ``weaponize'' congestion against former allies.

\subsection{Mechanism of Coalition Splitting}

We next investigate the precise mechanism by which spite induces grand coalition collapse. Fixing $\rho = 0.0005$ (mild altruism), we sweep $\sigma$ from 0 to 0.01 and measure the \emph{split incentive}---the utility gain from the most profitable binary split of the grand coalition.

\begin{figure}[htbp]
\centering
\includegraphics[width=0.85\textwidth]{figures/splitting_mechanism.png}
\caption{Mechanism of grand coalition collapse. Top panel: Split incentive $\Delta V$ increases monotonically with $\sigma$. Bottom panel: Social cost remains constant at 17.0, demonstrating that splitting is motivated by strategic spite, not efficiency concerns.}
\label{fig:splitting_mechanism}
\end{figure}

Figure~\ref{fig:splitting_mechanism} reveals that the split incentive grows linearly with $\sigma$, from approximately 0.002 at $\sigma = 0$ to 0.204 at $\sigma = 0.01$. Crucially, the social cost remains constant at 17.0 throughout this range. This demonstrates a key insight: \emph{coalition splitting is not driven by efficiency gains, but by the strategic value of creating out-groups to harm}.

Under the normalized spite term $\sigma \cdot \bar{\mu}_{\mathrm{out}}$, splitting the grand coalition creates a non-empty out-group, enabling players to derive positive utility from their rivals' suffering. This ``weaponization'' of congestion represents a novel form of strategic behavior that emerges only when social preferences interact with coalition structure.

\subsection{Price of Stability}

We assess the efficiency implications of stable coalition structures by computing the \emph{price of stability}---the ratio of social cost at the stable structure's equilibrium to the selfish baseline cost.

\begin{figure}[htbp]
\centering
\includegraphics[width=0.95\textwidth]{figures/price_of_stability.png}
\caption{Price of stability analysis. Left: Cost ratio heatmap showing that stable structures (predominantly singletons) achieve costs within 3\% of the selfish baseline. Right: The singleton structure dominates across nearly all parameter combinations.}
\label{fig:price_of_stability}
\end{figure}

Figure~\ref{fig:price_of_stability} shows that the average cost ratio in the nearly-selfish regime is 0.981, indicating a modest 1.9\% efficiency improvement over the selfish baseline. However, this improvement is almost entirely attributable to the grand coalition at the origin. Across the broader parameter space, the singleton structure---which dominates stability---achieves social cost of 17.5 versus the baseline of 17.5, yielding a cost ratio of exactly 1.0.

This finding suggests that \emph{structural stability and efficiency are largely orthogonal under normalized social preferences}. The most stable structure (singleton) preserves selfish efficiency, while potentially more efficient structures (grand coalition) are highly unstable.

\subsection{Fragility of Generic Structures}

Finally, we test whether stability is a property of our chosen archetypes or a more general feature of the coalition landscape. We sample 100 random asymmetric coalition structures and test their stability at moderate parameters $(\rho, \sigma) = (0.0002, 0.0002)$.

\begin{figure}[htbp]
\centering
\includegraphics[width=0.85\textwidth]{figures/structure_fragility.png}
\caption{Fragility of generic coalition structures. Left: Only 3\% of random structures are stable, compared to 33\% of archetypes. Right: Failure mode breakdown shows split deviations (52.4\%) and combined split-join instability (43.7\%) dominate.}
\label{fig:structure_fragility}
\end{figure}

Figure~\ref{fig:structure_fragility} demonstrates extreme fragility: only 3\% of random structures achieve stability, compared to 33\% for archetypes (specifically, the singleton). The failure mode breakdown reveals that 52.4\% of structures fail due to profitable splits, 43.7\% fail due to both split and join deviations, and 0\% fail due to joins alone.

This asymmetry---splits dominate joins---reflects the normalized spite formulation. Splitting creates new out-group targets for spite-motivated harm, while joining dilutes both in-group altruism benefits and out-group targeting capability. The result is a strong evolutionary pressure toward atomization.

\subsection{Summary of Findings}

Our computational experiments yield four key insights:

\begin{enumerate}
    \item \textbf{Singleton dominance}: Under normalized social preferences, the singleton (fully atomized) structure is overwhelmingly stable (91.7\%), while cooperative structures are extremely fragile (0.8\%).
    
    \item \textbf{Spite-driven splitting}: Coalition collapse is driven by the strategic value of creating out-groups, not by efficiency considerations. The split incentive scales linearly with spite $\sigma$.
    
    \item \textbf{Efficiency-stability orthogonality}: Stable structures (singletons) preserve selfish efficiency, while potentially more efficient structures (grand coalition) are unstable except at the selfish baseline.
    
    \item \textbf{Generic fragility}: Random asymmetric structures achieve only 3\% stability, with split deviations as the dominant failure mode. This suggests that extreme structures (full cooperation or full atomization) are the only robust outcomes.
\end{enumerate}

These findings highlight the destabilizing power of even small amounts of spite in coalition formation, and suggest that mechanism design in social networks must carefully account for the interaction between group identity and strategic preferences.

\section{Discussion}

\section{Conclusion}


\newpage
\printbibliography
\end{document}
